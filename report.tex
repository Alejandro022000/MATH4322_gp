% Options for packages loaded elsewhere
\PassOptionsToPackage{unicode}{hyperref}
\PassOptionsToPackage{hyphens}{url}
%
\documentclass[
]{article}
\usepackage{amsmath,amssymb}
\usepackage{iftex}
\ifPDFTeX
  \usepackage[T1]{fontenc}
  \usepackage[utf8]{inputenc}
  \usepackage{textcomp} % provide euro and other symbols
\else % if luatex or xetex
  \usepackage{unicode-math} % this also loads fontspec
  \defaultfontfeatures{Scale=MatchLowercase}
  \defaultfontfeatures[\rmfamily]{Ligatures=TeX,Scale=1}
\fi
\usepackage{lmodern}
\ifPDFTeX\else
  % xetex/luatex font selection
\fi
% Use upquote if available, for straight quotes in verbatim environments
\IfFileExists{upquote.sty}{\usepackage{upquote}}{}
\IfFileExists{microtype.sty}{% use microtype if available
  \usepackage[]{microtype}
  \UseMicrotypeSet[protrusion]{basicmath} % disable protrusion for tt fonts
}{}
\makeatletter
\@ifundefined{KOMAClassName}{% if non-KOMA class
  \IfFileExists{parskip.sty}{%
    \usepackage{parskip}
  }{% else
    \setlength{\parindent}{0pt}
    \setlength{\parskip}{6pt plus 2pt minus 1pt}}
}{% if KOMA class
  \KOMAoptions{parskip=half}}
\makeatother
\usepackage{xcolor}
\usepackage[margin=1in]{geometry}
\usepackage{color}
\usepackage{fancyvrb}
\newcommand{\VerbBar}{|}
\newcommand{\VERB}{\Verb[commandchars=\\\{\}]}
\DefineVerbatimEnvironment{Highlighting}{Verbatim}{commandchars=\\\{\}}
% Add ',fontsize=\small' for more characters per line
\usepackage{framed}
\definecolor{shadecolor}{RGB}{248,248,248}
\newenvironment{Shaded}{\begin{snugshade}}{\end{snugshade}}
\newcommand{\AlertTok}[1]{\textcolor[rgb]{0.94,0.16,0.16}{#1}}
\newcommand{\AnnotationTok}[1]{\textcolor[rgb]{0.56,0.35,0.01}{\textbf{\textit{#1}}}}
\newcommand{\AttributeTok}[1]{\textcolor[rgb]{0.77,0.63,0.00}{#1}}
\newcommand{\BaseNTok}[1]{\textcolor[rgb]{0.00,0.00,0.81}{#1}}
\newcommand{\BuiltInTok}[1]{#1}
\newcommand{\CharTok}[1]{\textcolor[rgb]{0.31,0.60,0.02}{#1}}
\newcommand{\CommentTok}[1]{\textcolor[rgb]{0.56,0.35,0.01}{\textit{#1}}}
\newcommand{\CommentVarTok}[1]{\textcolor[rgb]{0.56,0.35,0.01}{\textbf{\textit{#1}}}}
\newcommand{\ConstantTok}[1]{\textcolor[rgb]{0.00,0.00,0.00}{#1}}
\newcommand{\ControlFlowTok}[1]{\textcolor[rgb]{0.13,0.29,0.53}{\textbf{#1}}}
\newcommand{\DataTypeTok}[1]{\textcolor[rgb]{0.13,0.29,0.53}{#1}}
\newcommand{\DecValTok}[1]{\textcolor[rgb]{0.00,0.00,0.81}{#1}}
\newcommand{\DocumentationTok}[1]{\textcolor[rgb]{0.56,0.35,0.01}{\textbf{\textit{#1}}}}
\newcommand{\ErrorTok}[1]{\textcolor[rgb]{0.64,0.00,0.00}{\textbf{#1}}}
\newcommand{\ExtensionTok}[1]{#1}
\newcommand{\FloatTok}[1]{\textcolor[rgb]{0.00,0.00,0.81}{#1}}
\newcommand{\FunctionTok}[1]{\textcolor[rgb]{0.00,0.00,0.00}{#1}}
\newcommand{\ImportTok}[1]{#1}
\newcommand{\InformationTok}[1]{\textcolor[rgb]{0.56,0.35,0.01}{\textbf{\textit{#1}}}}
\newcommand{\KeywordTok}[1]{\textcolor[rgb]{0.13,0.29,0.53}{\textbf{#1}}}
\newcommand{\NormalTok}[1]{#1}
\newcommand{\OperatorTok}[1]{\textcolor[rgb]{0.81,0.36,0.00}{\textbf{#1}}}
\newcommand{\OtherTok}[1]{\textcolor[rgb]{0.56,0.35,0.01}{#1}}
\newcommand{\PreprocessorTok}[1]{\textcolor[rgb]{0.56,0.35,0.01}{\textit{#1}}}
\newcommand{\RegionMarkerTok}[1]{#1}
\newcommand{\SpecialCharTok}[1]{\textcolor[rgb]{0.00,0.00,0.00}{#1}}
\newcommand{\SpecialStringTok}[1]{\textcolor[rgb]{0.31,0.60,0.02}{#1}}
\newcommand{\StringTok}[1]{\textcolor[rgb]{0.31,0.60,0.02}{#1}}
\newcommand{\VariableTok}[1]{\textcolor[rgb]{0.00,0.00,0.00}{#1}}
\newcommand{\VerbatimStringTok}[1]{\textcolor[rgb]{0.31,0.60,0.02}{#1}}
\newcommand{\WarningTok}[1]{\textcolor[rgb]{0.56,0.35,0.01}{\textbf{\textit{#1}}}}
\usepackage{graphicx}
\makeatletter
\def\maxwidth{\ifdim\Gin@nat@width>\linewidth\linewidth\else\Gin@nat@width\fi}
\def\maxheight{\ifdim\Gin@nat@height>\textheight\textheight\else\Gin@nat@height\fi}
\makeatother
% Scale images if necessary, so that they will not overflow the page
% margins by default, and it is still possible to overwrite the defaults
% using explicit options in \includegraphics[width, height, ...]{}
\setkeys{Gin}{width=\maxwidth,height=\maxheight,keepaspectratio}
% Set default figure placement to htbp
\makeatletter
\def\fps@figure{htbp}
\makeatother
\setlength{\emergencystretch}{3em} % prevent overfull lines
\providecommand{\tightlist}{%
  \setlength{\itemsep}{0pt}\setlength{\parskip}{0pt}}
\setcounter{secnumdepth}{-\maxdimen} % remove section numbering
\ifLuaTeX
  \usepackage{selnolig}  % disable illegal ligatures
\fi
\IfFileExists{bookmark.sty}{\usepackage{bookmark}}{\usepackage{hyperref}}
\IfFileExists{xurl.sty}{\usepackage{xurl}}{} % add URL line breaks if available
\urlstyle{same}
\hypersetup{
  pdftitle={report},
  pdfauthor={Alejandro Perez},
  hidelinks,
  pdfcreator={LaTeX via pandoc}}

\title{report}
\author{Alejandro Perez}
\date{2023-04-17}

\begin{document}
\maketitle

\hypertarget{introduction}{%
\subsection{Introduction}\label{introduction}}

In this analysis, we use a publicly available white wine dataset to
answer the question: which factors contribute to a higher quality
rating? We will investigate whether higher acidity rates, pH, and
density rates affect the quality of the wine. The dataset contains 11
input features (fixed.acidity, volatile.acidity, citric.acid,
residual.sugar, chlorides, free.sulfur.dioxide, total.sulfur.dioxide,
density, pH, sulphates, alcohol) and one output variable (quality).

\hypertarget{methods}{%
\subsection{Methods}\label{methods}}

We use linear regression and random forest models to answer the
question. Linear regression is a simple and interpretable model, while
random forest is a more complex, non-linear model that can capture
intricate relationships between variables. Both models have their
advantages and disadvantages, and by comparing their performance, we can
gain insights into the underlying relationships between the predictors
and the response variable.

\begin{Shaded}
\begin{Highlighting}[]
\CommentTok{\# Load necessary libraries}
\FunctionTok{library}\NormalTok{(tidyverse)}
\end{Highlighting}
\end{Shaded}

\begin{verbatim}
## Warning: package 'tidyverse' was built under R version 4.1.3
\end{verbatim}

\begin{verbatim}
## Warning: package 'ggplot2' was built under R version 4.1.3
\end{verbatim}

\begin{verbatim}
## Warning: package 'tibble' was built under R version 4.1.3
\end{verbatim}

\begin{verbatim}
## Warning: package 'tidyr' was built under R version 4.1.3
\end{verbatim}

\begin{verbatim}
## Warning: package 'readr' was built under R version 4.1.3
\end{verbatim}

\begin{verbatim}
## Warning: package 'purrr' was built under R version 4.1.3
\end{verbatim}

\begin{verbatim}
## Warning: package 'dplyr' was built under R version 4.1.3
\end{verbatim}

\begin{verbatim}
## Warning: package 'stringr' was built under R version 4.1.3
\end{verbatim}

\begin{verbatim}
## Warning: package 'forcats' was built under R version 4.1.3
\end{verbatim}

\begin{verbatim}
## Warning: package 'lubridate' was built under R version 4.1.3
\end{verbatim}

\begin{verbatim}
## -- Attaching core tidyverse packages ------------------------ tidyverse 2.0.0 --
## v dplyr     1.1.0     v readr     2.1.4
## v forcats   1.0.0     v stringr   1.5.0
## v ggplot2   3.4.1     v tibble    3.2.1
## v lubridate 1.9.2     v tidyr     1.3.0
## v purrr     1.0.1     
## -- Conflicts ------------------------------------------ tidyverse_conflicts() --
## x dplyr::filter() masks stats::filter()
## x dplyr::lag()    masks stats::lag()
## i Use the ]8;;http://conflicted.r-lib.org/conflicted package]8;; to force all conflicts to become errors
\end{verbatim}

\begin{Shaded}
\begin{Highlighting}[]
\FunctionTok{library}\NormalTok{(caret)}
\end{Highlighting}
\end{Shaded}

\begin{verbatim}
## Warning: package 'caret' was built under R version 4.1.3
\end{verbatim}

\begin{verbatim}
## Loading required package: lattice
## 
## Attaching package: 'caret'
## 
## The following object is masked from 'package:purrr':
## 
##     lift
\end{verbatim}

\begin{Shaded}
\begin{Highlighting}[]
\FunctionTok{library}\NormalTok{(randomForest)}
\end{Highlighting}
\end{Shaded}

\begin{verbatim}
## Warning: package 'randomForest' was built under R version 4.1.3
\end{verbatim}

\begin{verbatim}
## randomForest 4.7-1.1
## Type rfNews() to see new features/changes/bug fixes.
## 
## Attaching package: 'randomForest'
## 
## The following object is masked from 'package:dplyr':
## 
##     combine
## 
## The following object is masked from 'package:ggplot2':
## 
##     margin
\end{verbatim}

\begin{Shaded}
\begin{Highlighting}[]
\CommentTok{\# Read the dataset}
\NormalTok{url }\OtherTok{\textless{}{-}} \StringTok{"https://archive.ics.uci.edu/ml/machine{-}learning{-}databases/wine{-}quality/winequality{-}white.csv"}
\NormalTok{white\_wine }\OtherTok{\textless{}{-}} \FunctionTok{read\_delim}\NormalTok{(url, }\AttributeTok{col\_names =} \ConstantTok{TRUE}\NormalTok{, }\AttributeTok{delim =} \StringTok{";"}\NormalTok{)}
\end{Highlighting}
\end{Shaded}

\begin{verbatim}
## Rows: 4898 Columns: 12
## -- Column specification --------------------------------------------------------
## Delimiter: ";"
## dbl (12): fixed acidity, volatile acidity, citric acid, residual sugar, chlo...
## 
## i Use `spec()` to retrieve the full column specification for this data.
## i Specify the column types or set `show_col_types = FALSE` to quiet this message.
\end{verbatim}

\begin{Shaded}
\begin{Highlighting}[]
\CommentTok{\# Replace spaces in column names with underscores}
\FunctionTok{colnames}\NormalTok{(white\_wine) }\OtherTok{\textless{}{-}} \FunctionTok{make.names}\NormalTok{(}\FunctionTok{colnames}\NormalTok{(white\_wine), }\AttributeTok{unique =} \ConstantTok{TRUE}\NormalTok{)}
\CommentTok{\# Set seed for reproducibility}
\FunctionTok{set.seed}\NormalTok{(}\DecValTok{123}\NormalTok{)}

\CommentTok{\# Train/Test split}
\NormalTok{train\_idx }\OtherTok{\textless{}{-}} \FunctionTok{createDataPartition}\NormalTok{(white\_wine}\SpecialCharTok{$}\NormalTok{quality, }\AttributeTok{p =} \FloatTok{0.8}\NormalTok{, }\AttributeTok{list =} \ConstantTok{FALSE}\NormalTok{)}
\NormalTok{train\_data }\OtherTok{\textless{}{-}}\NormalTok{ white\_wine[train\_idx, ]}
\NormalTok{test\_data }\OtherTok{\textless{}{-}}\NormalTok{ white\_wine[}\SpecialCharTok{{-}}\NormalTok{train\_idx, ]}
\CommentTok{\# Linear Regression}
\NormalTok{lm\_model }\OtherTok{\textless{}{-}} \FunctionTok{lm}\NormalTok{(quality }\SpecialCharTok{\textasciitilde{}}\NormalTok{ ., }\AttributeTok{data =}\NormalTok{ train\_data)}
\NormalTok{lm\_summary }\OtherTok{\textless{}{-}} \FunctionTok{summary}\NormalTok{(lm\_model)}

\CommentTok{\# Random Forest}
\NormalTok{rf\_model }\OtherTok{\textless{}{-}} \FunctionTok{randomForest}\NormalTok{(quality }\SpecialCharTok{\textasciitilde{}}\NormalTok{ ., }\AttributeTok{data =}\NormalTok{ train\_data)}
\CommentTok{\# Evaluate models on test data}
\NormalTok{lm\_pred }\OtherTok{\textless{}{-}} \FunctionTok{predict}\NormalTok{(lm\_model, test\_data)}
\NormalTok{lm\_rmse }\OtherTok{\textless{}{-}} \FunctionTok{sqrt}\NormalTok{(}\FunctionTok{mean}\NormalTok{((test\_data}\SpecialCharTok{$}\NormalTok{quality }\SpecialCharTok{{-}}\NormalTok{ lm\_pred)}\SpecialCharTok{\^{}}\DecValTok{2}\NormalTok{))}

\NormalTok{rf\_pred }\OtherTok{\textless{}{-}} \FunctionTok{predict}\NormalTok{(rf\_model, test\_data)}
\NormalTok{rf\_rmse }\OtherTok{\textless{}{-}} \FunctionTok{sqrt}\NormalTok{(}\FunctionTok{mean}\NormalTok{((test\_data}\SpecialCharTok{$}\NormalTok{quality }\SpecialCharTok{{-}}\NormalTok{ rf\_pred)}\SpecialCharTok{\^{}}\DecValTok{2}\NormalTok{))}

\CommentTok{\# Variable Importance for Random Forest}
\NormalTok{var\_importance }\OtherTok{\textless{}{-}} \FunctionTok{importance}\NormalTok{(rf\_model)}
\NormalTok{var\_importance\_plot }\OtherTok{\textless{}{-}} \FunctionTok{varImpPlot}\NormalTok{(rf\_model)}
\end{Highlighting}
\end{Shaded}

\includegraphics{report_files/figure-latex/unnamed-chunk-2-1.pdf}

\begin{Shaded}
\begin{Highlighting}[]
\NormalTok{lm\_summary}
\end{Highlighting}
\end{Shaded}

\begin{verbatim}
## 
## Call:
## lm(formula = quality ~ ., data = train_data)
## 
## Residuals:
##     Min      1Q  Median      3Q     Max 
## -3.7087 -0.4905 -0.0320  0.4547  3.1181 
## 
## Coefficients:
##                        Estimate Std. Error t value Pr(>|t|)    
## (Intercept)           1.444e+02  2.024e+01   7.132 1.17e-12 ***
## fixed.acidity         5.502e-02  2.289e-02   2.403 0.016287 *  
## volatile.acidity     -1.890e+00  1.265e-01 -14.938  < 2e-16 ***
## citric.acid           6.838e-02  1.084e-01   0.631 0.528016    
## residual.sugar        7.833e-02  8.174e-03   9.583  < 2e-16 ***
## chlorides            -5.202e-01  5.981e-01  -0.870 0.384529    
## free.sulfur.dioxide   3.343e-03  9.369e-04   3.569 0.000363 ***
## total.sulfur.dioxide -3.570e-04  4.220e-04  -0.846 0.397604    
## density              -1.442e+02  2.054e+01  -7.019 2.62e-12 ***
## pH                    6.342e-01  1.162e-01   5.458 5.11e-08 ***
## sulphates             5.995e-01  1.130e-01   5.308 1.17e-07 ***
## alcohol               1.968e-01  2.623e-02   7.503 7.68e-14 ***
## ---
## Signif. codes:  0 '***' 0.001 '**' 0.01 '*' 0.05 '.' 0.1 ' ' 1
## 
## Residual standard error: 0.7499 on 3907 degrees of freedom
## Multiple R-squared:  0.2826, Adjusted R-squared:  0.2806 
## F-statistic: 139.9 on 11 and 3907 DF,  p-value: < 2.2e-16
\end{verbatim}

\begin{Shaded}
\begin{Highlighting}[]
\NormalTok{rf\_model}
\end{Highlighting}
\end{Shaded}

\begin{verbatim}
## 
## Call:
##  randomForest(formula = quality ~ ., data = train_data) 
##                Type of random forest: regression
##                      Number of trees: 500
## No. of variables tried at each split: 3
## 
##           Mean of squared residuals: 0.3668714
##                     % Var explained: 53.06
\end{verbatim}

\begin{Shaded}
\begin{Highlighting}[]
\FunctionTok{print}\NormalTok{(var\_importance\_plot)}
\end{Highlighting}
\end{Shaded}

\begin{verbatim}
##                      IncNodePurity
## fixed.acidity             184.2600
## volatile.acidity          315.3986
## citric.acid               192.4062
## residual.sugar            219.0714
## chlorides                 244.4757
## free.sulfur.dioxide       302.9807
## total.sulfur.dioxide      238.2083
## density                   337.4709
## pH                        203.5677
## sulphates                 175.9536
## alcohol                   494.5513
\end{verbatim}

Interpretation and Conclusion Based on the model summaries and RMSE
values, the random forest model (RMSE = r rf\_rmse) outperforms the
linear regression model (RMSE = r lm\_rmse) in predicting wine quality.
This suggests that the relationship between the predictors and wine
quality is not strictly linear, and the random forest model can capture
these more complex relationships.

The variable importance plot for the random forest model indicates that
alcohol content is the most important predictor of wine quality,
followed by density, volatile acidity, and free sulfur dioxide.

In the linear regression model, we observe the following relationships
with wine quality:

Higher fixed acidity has a small positive effect on quality (p
\textless{} 0.05). Higher volatile acidity has a negative effect on
quality (p \textless{} 0.001). Citric acid does not have a significant
effect on quality (p \textgreater{} 0.05). Higher residual sugar has a
positive effect on quality (p \textless{} 0.001). Chlorides do not have
a significant effect on quality (p \textgreater{} 0.05). Higher free
sulfur dioxide has a positive effect on quality (p \textless{} 0.001).
Total sulfur dioxide does not have a significant effect on quality (p
\textgreater{} 0.05). Higher density has a negative effect on quality (p
\textless{} 0.001). Higher pH has a positive effect on quality (p
\textless{} 0.001). Higher sulphates have a positive effect on quality
(p \textless{} 0.001). Higher alcohol content has a positive effect on
quality (p \textless{} 0.001). To summarize, the random forest model
shows that alcohol content, density, volatile acidity, and free sulfur
dioxide are the most important factors affecting wine quality. The
linear regression model largely agrees with these findings, but it also
identifies other significant predictors, such as fixed acidity, residual
sugar, pH, and sulphates. It is important to note that some of the
predictors' effects on wine quality might be nonlinear, and the random
forest model can better capture these relationships compared to the
linear regression model.

In conclusion, our analysis suggests that alcohol content is the most
influential factor in determining wine quality, while other factors such
as density, volatile acidity, and free sulfur dioxide also play
important roles. To improve wine quality, wine producers should focus on
optimizing these factors, keeping in mind that relationships between
these predictors and wine quality might be complex and nonlinear.

\end{document}
